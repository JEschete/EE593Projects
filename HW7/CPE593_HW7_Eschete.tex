\documentclass[12pt]{article}
\usepackage[margin=1in]{geometry}
\usepackage{amsmath,amssymb}
\setlength{\parskip}{1em}

\title{\textbf{CPE - 593: Homework 7 - SEQ - Huffman Trees and Hashing}}
\author{Jude M. Eschete}
\date{\today}

\begin{document}
	\maketitle
	
	\section*{1. Tree Applications}
	
	\subsection*{1.1 Pseudocode for Building a Huffman Tree (3 points)}
	\noindent \textbf{Goal:} Construct a Huffman tree for a given string based on the frequency of each character.\\[6pt]
	\noindent \textbf{Pseudocode:}
	\begin{enumerate}
		\item \textbf{Compute frequencies:} For each unique character in the input, count its frequency. Store this in a map or dictionary \(\{\,\text{char} : \text{freq}\}\).
		\item \textbf{Initialize a min-priority queue (min-heap):} 
		\begin{itemize}
			\item For each character \(c\) with frequency \(f_c\), create a leaf node. 
			\item Insert each node into the min-heap, with priority = frequency.
		\end{itemize}
		\item \textbf{Build the tree:} While there is more than one node in the min-heap:
		\begin{enumerate}
			\item Extract the two nodes with the smallest frequencies (call them \(n_1\) and \(n_2\)).
			\item Create a new internal node \(n_{\text{new}}\) with:
			\[
			\text{freq}(n_{\text{new}}) = \text{freq}(n_1) + \text{freq}(n_2)
			\]
			\item Set \(n_1\) as the left child and \(n_2\) as the right child of \(n_{\text{new}}\).
			\item Insert \(n_{\text{new}}\) back into the min-heap.
		\end{enumerate}
		\item \textbf{Final node:} When only one node remains in the min-heap, that node is the root of the Huffman tree.
	\end{enumerate}
	
	\subsection*{1.2 Huffman Codes for \texttt{AAAABBCDDEEEEEE} (3 points)}
	We first compute the frequency of each character. For the string \(\texttt{AAAABBCDDEEEEEE}\):
	\[
	\begin{aligned}
		A &: 4,\\
		B &: 2,\\
		C &: 1,\\
		D &: 2,\\
		E &: 6.
	\end{aligned}
	\]
	We then build the Huffman tree by combining the lowest-frequency nodes first. A possible final Huffman tree (one of equally valid variations) assigns the codes:
	\[
	\begin{aligned}
		A &\rightarrow 10,\\
		B &\rightarrow 1111,\\
		C &\rightarrow 1110,\\
		D &\rightarrow 110,\\
		E &\rightarrow 0.
	\end{aligned}
	\]
	
	\subsection*{1.3 Encoding of the Original String (2 points)}
	Using the codes above, we convert each character of \(\texttt{AAAABBCDDEEEEEE}\) into its Huffman bit string:
	
	\[
	\texttt{A} = 10,\quad
	\texttt{B} = 1111,\quad
	\texttt{C} = 1110,\quad
	\texttt{D} = 110,\quad
	\texttt{E} = 0.
	\]
	
	\noindent The original string is 
	\[
	\underbrace{\texttt{AAAA}}_{4\times A}\,
	\underbrace{\texttt{BB}}_{2\times B}\,
	\texttt{C}\,
	\underbrace{\texttt{DD}}_{2\times D}\,
	\underbrace{\texttt{EEEEEE}}_{6\times E}.
	\]
	
	\noindent Substituting codes in order:
	\[
	\begin{aligned}
		\texttt{AAAA} &\rightarrow 10\,10\,10\,10 = 10101010,\\
		\texttt{BB}  &\rightarrow 1111\,1111 = 11111111,\\
		\texttt{C}    &\rightarrow 1110,\\
		\texttt{DD}  &\rightarrow 110\,110 = 110110,\\
		\texttt{EEEEEE} &\rightarrow 0\,0\,0\,0\,0\,0 = 000000.
	\end{aligned}
	\]
	
	\noindent Concatenating these segments, the final encoded bit string is:
	\[
	\boxed{10101010111111111110110110000000}.
	\]
	
	\section*{2. True or False}
	
	\subsection*{2.1 (1 point)}
	\emph{``A hash function may give the same hash value for distinct input data.''}
	
	\noindent This statement is \(\boxed{\text{True}}\), since different keys can naturally collide to the same hash value.
	
	\subsection*{2.2 (1 point)}
	\emph{``When the load factor exceeds a threshold, rehashing is performed. A key stored in the original hash table is copied to the bin of the same index in the new hash table.''}
	
	\noindent This statement is \(\boxed{\text{False}}\). Upon rehashing (especially if the table size changes), each key’s hash value is recomputed. Typically, items do not remain in the same index after resizing; they are redistributed based on the new table size.
	
	\newpage 
	
	\section*{3. Hashing Algorithms (8 points)}
	
	Below, we use the hash function \(\text{hash}(x) = x \bmod 16\) and insert the keys 
	\[
	\{36,\,52,\,30,\,22,\,38,\,54\}.
	\]
	We illustrate four collision-resolution strategies: 
	\emph{(a) linear probing}, 
	\emph{(b) quadratic probing}, 
	\emph{(c) double hashing}, 
	and 
	\emph{(d) separate chaining}. 
	In each case, we track the final position (or chain) of each key.
	
	\subsection*{3.1 m=16, Linear Probing}
	\noindent
	\textbf{Formula Setup:} 
	\[
	\text{hash}(x) = x \bmod 16, 
	\quad 
	\text{Probe sequence: }h(x) + i \quad (i = 0,1,2,\dots).
	\]
	
	\noindent
	\textbf{Insertion order:} \(36,\,52,\,30,\,22,\,38,\,54\). 
	\begin{enumerate}
		\item \(\text{hash}(36) = 4\). Slot 4 is empty, place \(36\) at index 4.
		\item \(\text{hash}(52) = 4\). Collision at index 4; probe index 5 (empty), place \(52\) at index 5.
		\item \(\text{hash}(30) = 14\). Slot 14 is empty, place \(30\) at index 14.
		\item \(\text{hash}(22) = 6\). Slot 6 is empty, place \(22\) at index 6.
		\item \(\text{hash}(38) = 6\). Collision at index 6; probe index 7 (empty), place \(38\) at index 7.
		\item \(\text{hash}(54) = 6\). Collision at index 6, then 7; probe index 8 (empty), place \(54\) at index 8.
	\end{enumerate}
	
	\[
	\begin{array}{c|cccccccccccccccc}
		\text{Index} & 0 & 1 & 2 & 3 & 4 & 5 & 6 & 7 & 8 & 9 & 10 & 11 & 12 & 13 & 14 & 15 \\
		\hline
		\text{Value} & - & - & - & - & 36 & 52 & 22 & 38 & 54 & - & - & - & - & - & 30 & -
	\end{array}
	\]
	
	\subsection*{3.2 m=16, Quadratic ``Chaining'' (Quadratic Probing)}
	\noindent
	\textbf{Formula Setup:} 
	\[
	\text{hash}(x) = x \bmod 16, 
	\quad 
	\text{Probe sequence: } h(x) + i^2 \quad (i = 0,1,2,\dots).
	\]
	
	\noindent
	\textbf{Insertion order:} \(36,\,52,\,30,\,22,\,38,\,54\). 
	\begin{enumerate}
		\item \(36 \bmod 16 = 4.\) Place \(36\) at index 4.
		\item \(52 \bmod 16 = 4.\) Collision at 4; try \(4 + 1^2 = 5\) (empty), place \(52\) at 5.
		\item \(30 \bmod 16 = 14.\) Place \(30\) at index 14.
		\item \(22 \bmod 16 = 6.\) Place \(22\) at index 6.
		\item \(38 \bmod 16 = 6.\) Collision at 6; try \(6 + 1^2 = 7\) (empty), place \(38\) at 7.
		\item \(54 \bmod 16 = 6.\) Collision at 6 and 7; try \(6 + 1^2 = 7\) (taken), then \(6 + 2^2 = 10\) (empty), place \(54\) at 10.
	\end{enumerate}
	
	\[
	\begin{array}{c|cccccccccccccccc}
		\text{Index} & 0 & 1 & 2 & 3 & 4 & 5 & 6 & 7 & 8 & 9 & 10 & 11 & 12 & 13 & 14 & 15 \\
		\hline
		\text{Value} & - & - & - & - & 36 & 52 & 22 & 38 & - & - & 54 & - & - & - & 30 & -
	\end{array}
	\]
	
	\subsection*{3.3 m=16, Double Hashing}
	\noindent
	\textbf{Formula Setup:} 
	\[
	\text{hash}_1(x) = x \bmod 16, 
	\quad
	\text{hash}_2(x) = 7 - (x \bmod 7),
	\]
	\[
	\text{Probe sequence: } h_1(x) + i \cdot h_2(x) \quad (i = 0,1,2,\dots).
	\]
	
	\noindent
	\textbf{Insertion order:} \(36,\,52,\,30,\,22,\,38,\,54\). 
	\begin{enumerate}
		\item \(36\): 
		\(\text{hash}_1(36) = 4\). 
		Place \(36\) at index 4.
		\item \(52\): 
		\(\text{hash}_1(52) = 4\) (collision), 
		\(\text{hash}_2(52) = 7 - (52 \bmod 7) = 7 - 3 = 4.\)
		Next index = \((4 + 1\times 4) \bmod 16 = 8.\) Slot 8 empty, place \(52\).
		\item \(30\): 
		\(\text{hash}_1(30) = 14\). 
		Place \(30\) at index 14.
		\item \(22\): 
		\(\text{hash}_1(22) = 6\). 
		Slot 6 empty, place \(22\).
		\item \(38\): 
		\(\text{hash}_1(38) = 6\) (collision), 
		\(\text{hash}_2(38) = 7 - (38 \bmod 7) = 7 - 3 = 4.\)
		Next index = \((6 + 1\times 4) \bmod 16 = 10.\) Slot 10 empty, place \(38\).
		\item \(54\): 
		\(\text{hash}_1(54) = 6\) (collision), 
		\(\text{hash}_2(54) = 7 - (54 \bmod 7) = 7 - 5 = 2.\)
		We check:
		\[
		(6 + 1\times 2) \bmod 16 = 8 \quad (\text{occupied by }52),
		\]
		\[
		(6 + 2\times 2) \bmod 16 = 10 \quad (\text{occupied by }38),
		\]
		\[
		(6 + 3\times 2) \bmod 16 = 12 \quad (\text{empty}).
		\]
		Place \(54\) at index 12.
	\end{enumerate}
	
	\[
	\begin{array}{c|cccccccccccccccc}
		\text{Index} & 0 & 1 & 2 & 3 & 4 & 5 & 6 & 7 & 8 & 9 & 10 & 11 & 12 & 13 & 14 & 15 \\
		\hline
		\text{Value} & - & - & - & - & 36 & - & 22 & - & 52 & - & 38 & - & 54 & - & 30 & -
	\end{array}
	\]
	
	\subsection*{3.4 m=16, Separate Linear Chaining}
	\noindent
	\textbf{Formula Setup:} 
	\[
	\text{hash}(x) = x \bmod 16;
	\quad
	\text{Collisions in the same bin using a linear (linked) chain}.
	\]
	
	\noindent
	\textbf{Insertion order:} \(36,\,52,\,30,\,22,\,38,\,54\). 
	\begin{enumerate}
		\item \(\text{hash}(36) = 4.\) Bucket 4: [\,36\,].
		\item \(\text{hash}(52) = 4.\) Bucket 4: [\,36, 52\,].
		\item \(\text{hash}(30) = 14.\) Bucket 14: [\,30\,].
		\item \(\text{hash}(22) = 6.\) Bucket 6: [\,22\,].
		\item \(\text{hash}(38) = 6.\) Bucket 6: [\,22, 38\,].
		\item \(\text{hash}(54) = 6.\) Bucket 6: [\,22, 38, 54\,].
	\end{enumerate}
	
	\[
	\begin{array}{c|l}
		\text{Index} & \text{Chain} \\
		\hline
		0 & [\,] \\
		1 & [\,] \\
		2 & [\,] \\
		3 & [\,] \\
		4 & [\,36,\,52\,] \\
		5 & [\,] \\
		6 & [\,22,\,38,\,54\,] \\
		7 & [\,] \\
		8 & [\,] \\
		9 & [\,] \\
		10 & [\,] \\
		11 & [\,] \\
		12 & [\,] \\
		13 & [\,] \\
		14 & [\,30\,] \\
		15 & [\,] \\
	\end{array}
	\]
	
	\vspace{1em}
	\noindent
	\textbf{Summary:} 
	Each strategy successfully inserts all keys, but the collision-handling differs. 
	\emph{Linear probing} clusters keys contiguously, 
	\emph{Quadratic probing} uses a square-step offset, 
	\emph{Double hashing} incorporates a second hash function for increments, 
	and 
	\emph{Separate chaining} places colliding keys in a linked list at each index.
	
	\section*{4. Self-Study on Java 17 \texttt{Hashtable} Implementation (2 points)}
	
	\paragraph{}In Java 17, the \texttt{Hashtable} class is a legacy collection introduced in the earliest versions of Java and remains for historical and backward-compatibility reasons. Unlike the more commonly used \texttt{HashMap}, \texttt{Hashtable} is synchronized by default, making it thread-safe in simple concurrent scenarios. This synchronization may introduce performance overhead, which is why many modern applications favor using \texttt{HashMap} (often in combination with a \texttt{Collections.synchronizedMap} wrapper or a \texttt{ConcurrentHashMap} for robust concurrency).
	
	\paragraph{}Internally, \texttt{Hashtable} uses \textbf{separate chaining} to resolve collisions. Specifically, it maintains an array of buckets, each of which stores a singly linked list (a chain) of \texttt{Entry} objects. An \texttt{Entry} contains the key, the associated value, the key’s hash code, and a pointer to the next \texttt{Entry} in the chain. When new elements are inserted, \(\text{hash}(key) \bmod \text{capacity}\) determines the bucket index. If the bucket is already occupied, the element is linked after existing nodes. This design ensures that multiple keys mapping to the same index do not overwrite each other but simply join the existing chain.
	
	\paragraph{}If the number of stored entries grows large relative to the table size, the load factor triggers a rehash. During rehashing, \texttt{Hashtable} doubles its capacity (unless specified otherwise) and recomputes the bucket for each key. This process distributes the keys more evenly across the enlarged table, preventing excessively long chains from forming. Despite this automatic resizing, high contention or extremely large data sets can still degrade performance due to the linear traversal of chains.
	
	\paragraph{}Overall, Java 17’s \texttt{Hashtable} adheres to a traditional collision-handling approach by linking entries in a chain. While it may suffice for smaller-scale, concurrent usage, developers often choose \texttt{HashMap} for non-synchronized contexts. For heavily multi-threaded environments, \texttt{ConcurrentHashMap} is often a more efficient choice, since it avoids locking the entire map during updates. Nevertheless, the under-the-hood mechanism for dealing with collisions—separate chaining—has remained consistent within \texttt{Hashtable} since its inception.
\end{document}